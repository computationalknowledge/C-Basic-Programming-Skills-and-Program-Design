\documentclass{article}
\usepackage{graphics}
\usepackage{url}
\usepackage[utf8]{inputenc}
\usepackage{enumerate}
\usepackage[checklist]{enumitem}

\renewcommand{\labelitemi}{$\Box$}
\renewcommand{\labelitemi}{$\star$}
\renewcommand{\labelenumi}{\theenumi}

\usepackage{enumitem,amssymb}
\newlist{todolist}{itemize}{2}
\setlist[todolist]{label=$\square$}

\usepackage{geometry}
 \geometry{
 a4paper,
 total={170mm,257mm},
 left=20mm,
 top=20mm,
 }
% https://ipfs-sec.stackexchange.cloudflare-ipfs.com/tex/A/question/48036.html

\title{C\# Basic Programming Skills and Program Design}
\author{Instructor: Peter Sigurdson }
\date{August 2020}

\begin{document}

\maketitle

\section * {Core Skills Implementing Algorithms and Program Design}
    \begin{todolist}
      \item Sublist item 1 goes here.
      \item Sublist item 2 goes here.
    \end{todolist}

  \begin{itemize}[label={$\bullet$}]

    \item    WHY do we construct software applications?
    \item      To implement Data Storage and Rules Processing  requirements [Data storage + Rules Process = Algorithm]
    \item        We write Software to implement Algorithms
    \item    Draw pictures to Visualize the data storage and processing in a Software Application
    \item      What are the 2 ways we can construct communities of objects?
    \subitem        Inheritance
    \subsubitem          is-a relationship
    \subitem        Composition
    \subsubitem          has-a relationship
    \item    Be able to describe the advantages of using a SQL Data Store rather than relying on in-program variables and data structures
    \item Constructing a simple object-oriented program using C\#
    \item   Program Design using Classes and Objects to implement business logic
    \item     We write programs to simulate and automate (document automation for example) business processes and systems)
    \item     We write programs to do data analysis: charts, graphs, tables to visualize and summarize complex data
    \item   Objects communicate using Parameters and Return Types
    \item   The 4 Types of Variables
    \item     class Instance Variables
    \item       every OBJECT instantiated from that CLASS will have its own data
    \item     class static variables
    \item       ALL OBJECTS of that CLASS SHARE the same data
    \item     Method Local variables
    \item       that variable is SCOPED with the method; outside the method, nobody can see it
    \item     Loop Local Variables
    \item       that variable is SCOPED with a LOOP : IT IS NOT Available or Visible OUTSIDE of the loop

\end{itemize}

\end{document}
